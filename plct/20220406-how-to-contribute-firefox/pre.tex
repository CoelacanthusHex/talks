\documentclass[UTF-8]{ctexbeamer}
\usetheme{Berkeley}
\usecolortheme{seahorse}

\usepackage{minted}
\usepackage{tikz}
\usetikzlibrary{arrows,positioning,calc,decorations.pathreplacing,fit}
\usepackage{xcolor}
\usepackage{hyperref}
\usepackage{textcomp}
\usepackage{amsmath}
\usepackage{amssymb}
\usepackage{unicode-math}
\usepackage{graphicx}
\usepackage[normalem]{ulem}
\usepackage{tabularx}
\usepackage{multirow}
\usepackage{chronosys}

\setminted{
	tabsize=4,
	frame=lines
}
\hypersetup{
	colorlinks,
	linkcolor=blue,
	urlcolor=blue,
	anchorcolor=blue,
	citecolor=blue,
	breaklinks=true,
	hyperfootnotes=true,
	unicode
}

\newcommand{\emptyline}{\vspace{\baselineskip}}

\usepackage{FiraMono}

%\setmonofont{Fira Mono}
\setmainfont{Noto Serif}
\setsansfont{Noto Sans}
\setCJKmainfont{Noto Serif CJK SC}
\setCJKsansfont{Noto Sans CJK SC}
\setCJKmonofont{Noto Sans CJK SC}

\title{如何给火狐贡献代码}
%\subtitle{}
\author{Coelacanthus}
\institute{PLCT Arch RISC-V 小队}
\date{2022年4月6日}
\logo{
	\includegraphics[scale=0.75]{asserts/Archlinux-icon.pdf}
}

\setbeamertemplate{itemize items}[circle]
\setbeamertemplate{enumerate items}[default]
\beamertemplatenavigationsymbolsempty
\setbeamerfont{footnote}{size=\tiny}

%\urlstyle{tt}

\begin{document}

\frame{\titlepage}

\section{什么是火狐}
\subsection{工具链}
\begin{frame}
	\frametitle{Firefox 的维护工具链}

	\begin{itemize}%[<+->]
		\item 使用 C++ 和 Rust 开发
		\item 使用自有构建工具 Mach(由 Python 编写)(事实上具体的编译步骤会被托管到 Autotools Cargo 等具体的构建工具)
		\item 使用 Mercurial 作为 VCS(版本控制系统)
		\item 使用 Phabricator 作为代码审阅工具(Facebook 开发,被 Mozilla、GNUPG、KDE 和 LLVM 等大型项目广泛使用)
		\item 使用 Bugzilla 作为 bug 跟踪系统(Bugzilla 亦是 Mozilla 开发)
	\end{itemize}

\end{frame}

\subsection{维护架构}
\begin{frame}
	\frametitle{Firefox 的发行周期}

	\begin{itemize}
		\item 每四周发行一个版本
		\item Beta 和 Developer Edition 比 Release 快一个版本,Nightly 比 Beta 快一个版本
		\item 特殊挑选的 ESR 版本
	\end{itemize}

\end{frame}

\begin{frame}
	\frametitle{Firefox 的代码周期}

	\begin{itemize}
		\item 合并的代码会被加入 autoland 分支
		\item 高权限维护者不定期(平均每天几次)将 autoland 分支上的修改合并到 mozilla-central
		\item mozilla-central 是主要的开发分支,Nightly 是其在某个时刻的构建
		\item 每四周,在 mozilla-central 更新版本号之前,mozilla-beta 从 mozilla-central 分支出来,这就是 Beta/Developer 版本的维护分支
		\item 再过四周,在 mozilla-beta 从 mozilla-central 拉取代码前,mozilla-release 从他分支出来,这就是 Release 版本的维护分支
	\end{itemize}

\end{frame}

\section{怎样贡献火狐}

\subsection{去哪里寻找火狐的代码}
\begin{frame}
	\frametitle{去哪里寻找火狐的代码}

	Firefox 的代码库使用 Mercurial 托管在 \url{https://hg.mozilla.org/mozilla-central},\\
	但是同时,我们也有一个 Git 代码库可用 \url{https://github.com/mozilla/gecko-dev},\\
	虽然这个代码库是只读的,但是无关紧要,我们并不需要直接推送代码。

\end{frame}

\subsection{去哪里寻找和提出问题}
\begin{frame}[fragile]
	\frametitle{去哪里寻找和提出问题}

	Bug Tracker: \url{https://bugzilla.mozilla.org}
	\pause
	\emptyline
	点开 \href{https://bugzilla.mozilla.org/describecomponents.cgi}{Components} 可以看到 Firefox 的架构有数个部分,我们一般比较关心的是 Firefox、Core,其次是 DevTools 和 WebExtensions。注意 Firefox 的 Android 版并不在此追踪,他们使用 GitHub\footnote{\url{https://github.com/mozilla-mobile/fenix}}。

\end{frame}

\subsection{准备工具}
\begin{frame}[fragile]
	\frametitle{准备工具}

	当然,首先肯定需要 Python、C++ 编译器和 Rust 编译器,这是我们编译 Rust 所需要的基本工具。\\
	\pause
	然后,编译 Firefox 需要一些额外的工具,我并不会在这里为你列出需要的依赖,因为太多了。但是你可以通过以下方式获取:\\
	\pause
	\begin{description}
		\item[Arch] \href{https://github.com/archlinux/svntogit-packages/blob/packages/firefox/trunk/PKGBUILD#L13}{PKGBUILD of Firefox}
		\item[Fedora/RHEL] \verb|dnf builddep firefox|
		\item[Debian/Ubuntu] \verb|apt build-dep firefox|
	\end{description}
\end{frame}

\begin{frame}[fragile]
	\frametitle{准备工具}
	此外你需要安装一个名为 \verb|moz-phab| 的工具,可以通过 \verb|pip3 install MozPhab| 或者稍后通过 \verb|./mach install-moz-phab| 安装。\\
	或者你可以使用 arcanist 工具。关于这工具的使用方法,可以参考 \href{https://www.bilibili.com/video/BV14m4y1D7ZM}{刘阳 - 提交patch到LLVM上游} 和 \href{https://moz-conduit.readthedocs.io/en/latest/arcanist-user.html}{Arcanist User Guide} \\
	再或者你可以直接使用网页上传 patch,不推荐。\sout{不过有的用户,比如之前 LLVM 小队的 ksyx 就特别喜欢手动传 patch。}

\end{frame}

\begin{frame}[fragile]
	\frametitle{准备账号}

	首先,你需要有一个 BMO 账号,也就是 bugzilla.mozilla.org 的账号。\\
	其次,这个账号必须启用了二步验证。\\
	然后,你需要打开 \url{https://phabricator.services.mozilla.com/},完成注册。

\end{frame}

\subsection{修改与提交代码}
\subsubsection{获取代码库}
\begin{frame}[fragile]
	\frametitle{获取代码库 - Git}
	如果使用 Git 代码库,你需要安装一个名为 git-cinnabar 的辅助工具,安装方式在这里。 \url{https://github.com/glandium/git-cinnabar} \\

	然后,\verb|git clone https://github.com/mozilla/gecko-dev|。\\

	注意这个 Git 仓库有数个G,请保证你的网络环境稳定且高速率。

\end{frame}

\begin{frame}[fragile]
	\frametitle{获取代码库 - Mercurial}

	\verb|hg clone https://hg.mozilla.org/mozilla-central|

\end{frame}

\begin{frame}[fragile]
	\frametitle{获取代码库 - Official way}

	\begin{minted}[fontsize=\Tiny]{console}
$ curl -O https://hg.mozilla.org/mozilla-central/raw-file/default/python/mozboot/bin/bootstrap.py
$ python3 bootstrap.py
	\end{minted}
	如果使用 Git,如下,并且安装 git-cinnabar
	\begin{minted}[fontsize=\Tiny]{console}
$ curl -O https://hg.mozilla.org/mozilla-central/raw-file/default/python/mozboot/bin/bootstrap.py
$ python3 bootstrap.py --vcs=git
	\end{minted}
	这种情况下仓库在 mozilla-unified

\end{frame}

\subsubsection{修改与测试}
\begin{frame}[fragile]
	\frametitle{修改与测试}
	做你想做的修改,然后执行
	\begin{minted}{console}
$ ./mach build
$ ./mach test some/path
$ ./mach lint some/path
$ ./mach static-analysis
	\end{minted}
	然后你需要使用 \verb|./mach run| 运行火狐,检查功能是否正常工作。
\end{frame}

\subsubsection{提交代码}
\begin{frame}[fragile]
	\frametitle{提交代码}

	首先使用 \verb|moz-phab install-certificate| 完成账号配置,\\
	然后,一般情况下一个提交必须有对应的 Bug tracker id,所以你需要在 BMO 上创建一个跟踪器。\\
	使用 \verb|hg commit| 或者 \verb|git commit| 创建提交,格式如下\\
	\begin{minted}[fontsize=\tiny]{text}
Bug xxxx - Short description of your change. r=reviewer,#review-group

Optionally, a longer description of the change.
	\end{minted}
	Review Group 在这里可以找到:\url{https://firefox-source-docs.mozilla.org/contributing/reviews.html} \\
	你也可以查找版本历史寻找 Reviewer。\\
	然后使用 \verb|moz-phab| 提交代码。

\end{frame}

\subsection{请求审阅和更新提交}
\begin{frame}[fragile]
	\frametitle{审阅}
	然后,moz-phab 会返回一个链接,他用于追踪这次更改。\\
	你可以在网页上完成绝大多数操作。\\
	然后,等待 Reviewer 的到来。\\
	如果遇到有要求修改的地方,在本地修改,\verb|git commit --amend|,随后运行 \verb|moz-phab| 即可。
\end{frame}

\subsection{代码落地与后续跟踪}

\begin{frame}[fragile]
	\frametitle{后续跟踪}
	不要取消 Bugzilla 的 CC,有可能会有 regression 出现。
\end{frame}

\begin{frame}
	\frametitle{一些补充}
	
	\begin{itemize}
		\item try 如何下载编译产物 \href{https://bugzilla.mozilla.org/show\_bug.cgi?id=1756400}{演示链接}
	\end{itemize}
	
\end{frame}

\begin{frame}
	\begin{center}
		\huge{谢谢大家}
	\end{center}

\end{frame}

\end{document}
