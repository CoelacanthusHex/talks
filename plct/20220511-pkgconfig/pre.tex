\documentclass[UTF-8]{ctexbeamer}
\usetheme{Berkeley}
\usecolortheme{seahorse}

\usepackage{minted}
\usepackage{tikz}
\usetikzlibrary{arrows,positioning,calc,decorations.pathreplacing,fit}
\usepackage{xcolor}
\usepackage{hyperref}
\usepackage{textcomp}
\usepackage{amsmath}
\usepackage{amssymb}
\usepackage{unicode-math}
\usepackage{graphicx}
\usepackage[normalem]{ulem}
\usepackage{tabularx}
\usepackage{multirow}
\usepackage{chronosys}

\setminted{
	tabsize=4,
	frame=lines
}
\hypersetup{
	colorlinks,
	linkcolor=blue,
	urlcolor=blue,
	anchorcolor=blue,
	citecolor=blue,
	breaklinks=true,
	hyperfootnotes=true,
	unicode
}

\newcommand{\emptyline}{\vspace{\baselineskip}}

\usepackage{FiraMono}

%\setmonofont{Fira Mono}
\setmainfont{Noto Serif}
\setsansfont{Noto Sans}
\setCJKmainfont{Noto Serif CJK SC}
\setCJKsansfont{Noto Sans CJK SC}
\setCJKmonofont{Noto Sans CJK SC}

\title{Pkgconfig 的前世今生}
%\subtitle{}
\author{Coelacanthus}
\institute{PLCT Arch RISC-V 小队}
\date{2022年5月11日}
\logo{
	\includegraphics[scale=0.75]{assets/Archlinux-icon.pdf}
}

\setbeamertemplate{itemize items}[circle]
\setbeamertemplate{enumerate items}[default]
\beamertemplatenavigationsymbolsempty
\setbeamerfont{footnote}{size=\tiny}

%\urlstyle{tt}

\begin{document}

\frame{\titlepage}

\section{pkgconfig 是什么}
\begin{frame}
	\frametitle{pkgconfig 是什么}
	pkgconfig 是一个查询已安装库的信息的接口,由一个纯文本数据库和一个查询程序组成。
\end{frame}

\section{pkgconfig 的由来}
\begin{frame}
	\frametitle{pkgconfig 的由来}
	上世纪末,随着计算机软件的发展,可使用的库越来越多,人们迫切需要一个通用手段来得知可以用的库,并且方便的获取链接用的参数。\\
	更进一步的,人们想要它能够一定程度上处理依赖关系。\\
	于是在新世纪的第一年,James Henstridge 用 shell 编写了一个小程序和它对应的数据格式。\\
	当年七月份,为了更好的性能,Havoc Pennington 使用 C 和 Glib 库重写了 pkgconfig。
\end{frame}

\section{pkgconfig 的前辈}
\begin{frame}[fragile]
	\frametitle{pkgconfig 的前辈}
	事实上,在 1994 年,GNU 计划就编写了一个名为 GNU Libtool 的工具,用以处理不同平台上使用库的接口的统一性问题。
	但是在使用中,这个库表现出许多不甚满意的地方:
	\begin{itemize}
		\item libtool 是为和 Autoconf 和 Automake(也就是所谓的 Autotools 构建系统)配合编写的,与其他构建系统结合并不好
		\item libtool 使用 wrap 编译器的方式运作,侵入式很强,如果我们要使用 libtool,需要修改编译器的调用到如下形式 \verb|libtool gcc -o hello main.c libhello.la|
		\item libtool 只能和 libtool 配合使用,也就是说,应用和库必须都使用 libtool
		\item libtool 使用写死的路径,一旦路径改变,就需要重新处理
	\end{itemize}
\end{frame}

\section{pkgconfig 是怎么做的}
\begin{frame}
	\frametitle{pkgconfig 是怎么做的}

	\begin{itemize}
		\item pkgconfig 只是提供了一个查询的接口,用户如何使用具有很强的灵活性,也便于和构建系统结合
		\item 同样因为上一个原因,pkgconfig 几乎不具有侵入性,用户可以以他们想用的任何方式查询,然后传递给编译器
		\item 这样,它们可以和其他工具配合使用,用户只是需要把输出合并起来
		\item 使用标准的库查询路径机制,即使路径改变也只需要做很小的改动,甚至不需要改动
	\end{itemize}
\end{frame}

\section{pkgconfig 的机制}
\begin{frame}[fragile]
	\frametitle{pkgconfig 的机制}
	
	使用一个数据格式,称为 .pc,该文件格式如下
	\begin{minted}{text}
prefix=/usr
exec_prefix=${prefix}
includedir=${prefix}/include
libdir=${exec_prefix}/lib

Name: foo
Description: The foo library
Version: 2.1.2
Requires.private: bar >= 0.7
Requires: qwq >= 0.7
Conflicts: bar < 1.2.3, bar >= 1.3.0
Cflags: -I${includedir}
Libs: -L${libdir} -lbar
	\end{minted}
\end{frame}

\begin{frame}[fragile]
	\frametitle{pkgconfig 的机制}
	
	可以看出,该格式基本描述了链接一个库需要的全部信息:
	\begin{itemize}
		\item 这个库叫什么:这是我们查找的时候用的键值
		\item 这个库的描述:方便我们查看列表时判断功能
		\item 这个库的版本:用于兼容性判断
		\item 这个库的公开和私有依赖:依赖关系
		\item 这个库与什么相冲突:同样是依赖关系
		\item 传递给编译器和链接器的参数,这里同时处理不支持 pkgconfig 的依赖的依赖关系
	\end{itemize}
\end{frame}

\begin{frame}[fragile]
	\frametitle{pkgconfig 的机制}
	
	随后,你可以调用 pkgconfig 应用来查询这些数据,默认会查询 \verb|/usr/lib/pkgconfig|,同时你可以通过设定 \verb|PKG_CONFIG_PATH| 环境变量增加查询路径。\\
	
	一些常用的操作有:
	\begin{itemize}
		\item 列出所有已有的库:\verb|pkg-config --list-all|
		\item 获取某个库的 CFLAGS/CXXFLAGS:\verb|pkg-config --cflags xxxx|
		\item 获取某个库的 LDFLAGS:\verb|pkg-config --libs xxxx|
		\item 获取某个库静态链接的 LDFLAGS:\verb|pkg-config --static xxxx|
		\item 判断某个库是否存在:\verb|pkg-config --exists xxxx|
		\item 获取某个库的版本号:\verb|pkg-config --modversion xxxx|
	\end{itemize}
	注意此处查询参数可以写版本范围。
\end{frame}

\section{pkgconfig 的现代应用方式}
\begin{frame}[fragile]
	\frametitle{pkgconfig 的现代应用方式}
	
	几乎所有现代 C/C++ 构建系统都提供了对 pkgconfig 的包装:
	\begin{itemize}
		\item Autotools 可以使用通过 \verb|PKG_CHECK_EXISTS| 和 \verb|PKG_CHECK_MODULES| 宏操作
		\item CMake 可以使用 \verb|pkg_check_modules| 函数操作,并且相当大一部分 CMake FindModule 内部使用 pkgconfig 进行查询
		\item Meson 的 \verb|dependency| 会使用多种方法查找依赖,其中排在第一顺位的就是 pkgconfig\footnote{\url{https://mesonbuild.com/Dependencies.html}},同时 Meson 还提供了给项目生成 .pc 文件的简便方法\footnote{\url{https://mesonbuild.com/Pkgconfig-module.html}}
	\end{itemize}
\end{frame}

\begin{frame}
	\begin{center}
		\huge{谢谢大家}
	\end{center}

\end{frame}

\end{document}
