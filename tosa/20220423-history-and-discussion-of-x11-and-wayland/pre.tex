\documentclass[UTF-8]{ctexbeamer}
\usetheme{Berkeley}
\usecolortheme{seahorse}

\usepackage{minted}
\usepackage{tikz}
\usetikzlibrary{arrows,positioning,calc,decorations.pathreplacing,fit}
\usepackage[pdf]{graphviz}
\usepackage{xcolor}
\usepackage{hyperref}
\usepackage{textcomp}
\usepackage{amsmath}
\usepackage{amssymb}
\usepackage{unicode-math}
\usepackage{graphicx}
\usepackage[normalem]{ulem}
\usepackage{tabularx}
\usepackage{multirow}
\usepackage{chronosys}

\setminted{
	tabsize=4,
	frame=lines
}
\hypersetup{
	colorlinks,
	linkcolor=blue,
	urlcolor=blue,
	anchorcolor=blue,
	citecolor=blue,
	breaklinks=true,
	hyperfootnotes=true,
	unicode
}

\newcommand{\emptyline}{\vspace{\baselineskip}}

\usepackage{FiraMono}

%\setmonofont{Fira Mono}
\setmainfont{Noto Serif}
\setsansfont{Noto Sans}
\setCJKmainfont{Noto Serif CJK SC}
\setCJKsansfont{Noto Sans CJK SC}
\setCJKmonofont{Noto Sans CJK SC}

\title{X11 与 Wayland}
\subtitle{Linux GUI 的曲折发展史}
\author{Coelacanthus}
%\institute{PLCT Arch RISC-V 小队}
\date{2022年4月21日}
%\logo{
%	\includegraphics[scale=0.75]{assets/Archlinux-icon.pdf}
%}

\setbeamertemplate{itemize items}[circle]
\setbeamertemplate{enumerate items}[default]
\beamertemplatenavigationsymbolsempty
\setbeamerfont{footnote}{size=\tiny}

%\urlstyle{tt}

\begin{document}

\frame{\titlepage}

\section{X11 的由来}
\begin{frame}
	\frametitle{X11 的由来}
	MIT 造的图形界面轮子。全称是 X Window System。\\
	自1984年 X1 起步,到2012年 X11R7.7 发布。\\
	几个关键点:
	\begin{itemize}
		\item 1987年9月15日,现用协议 X11 的第一个版本。
		\item 1994年5月16日,X11R6,迁移到 XFree86 主持开发,xinput 等关键组件被添加。
		\item 2004年9月9日,X11R6.8.0,混成器拓展,已死的 XEvIE 拓展。
		\item 2007年2月15日,X11R7.2,XCB。
	\end{itemize}

\end{frame}

\section{X11 的基本架构}
\subsection{栈式窗口管理器}
\begin{frame}[fragile]
	\frametitle{栈式窗口管理器}
	早期栈式窗口管理器的设计特点:
	\begin{itemize}
		\item 使用画家算法绘制图像(或者说,深度缓冲技术?)
		\item 从而,可以重绘时仅绘制未被覆盖的部分,使得重绘永远不会超过屏幕面积
		\item 从而可以大幅度节省带宽和内存占用
	\end{itemize}
	
\end{frame}
\begin{frame}[fragile]
	\frametitle{早期栈式窗口管理器缺点}
	缺点:
	\begin{itemize}
		\item 窗口必须为矩形,否则无法处理
		\item 不支持半透明(毛玻璃爱好者无能狂怒)
		\item 为效率考虑,移动窗口的过程不会进行重绘
	\end{itemize}
	
	在早期 X,早期 Windows 和早期 mac 上都可以看到这些特点。
	
\end{frame}

\subsection{X11 的基本架构}
\begin{frame}[fragile]
	\frametitle{X11 的基本架构}
	
	\digraph[scale=0.32]{x11basic}{
		rankdir=LR;
		splines=ortho;
		node [shape=box];
		xlib [label="xlib/XCB"];
		gtk [label="GTK"];
		qt [label="Qt"];
		wm [label=<Window<br />Manager>];
		{rank=same; xlib; qt; gtk; wm;}
		pdf [label="Internal PDF Format"];
		qt_internal [label="Internal Format"];
		gtk_xpm [label="XPM"];
		qt_xpm [label="XPM"];
		xserver [label="Xserver", height=4];
		screen [label="Screen", height=2];
		keyboard [label="Keyboard", height=1];
		mouse [label="Mouse", height=1];
		{rank=same; screen; keyboard; mouse;}
		gtk -> pdf [label="Cairo"];
		pdf -> gtk_xpm;
		gtk_xpm -> xserver [label="xlib/xcb"];
		qt -> qt_internal [label="QPaint"];
		qt_internal -> qt_xpm;
		qt_xpm -> xserver [label="xlib/xcb"];
		xlib -> xserver [label="xlib/xcb"];
		wm -> xserver;
		xserver -> wm;
		gtk -> qt [color=white];
		qt -> xlib [color=white];
		xserver -> screen [label="KMS & DRM"];
		keyboard -> xserver [label=<evdev<br />and<br />xinput/libinput>];
		mouse -> xserver [label=<evdev<br />and<br />xinput/libinput>];
	}
	
\end{frame}

\section{混成器}

\begin{frame}[fragile]
	\frametitle{混成器拓展下的 X11 架构}
	为了提高绘图效率和完善功能,X11R6.8 引入了混成器拓展(Composite Extension)。
	主要内容是,提供 API,使得程序可以:
	\begin{itemize}
		\item 获取某个窗口的渲染结果(使用 \verb|RedirectSubwindows| 和 \verb|NameWindowPixmap| API)
		\item 得到一个特殊的绘图窗口,该窗口会覆盖在屏幕最上层(\verb|GetOverlayWindow|)
		\item 取得窗口裁剪边界(\verb|CreateRegionFromBorderClip|)
	\end{itemize}

\end{frame}

\begin{frame}[fragile]
	\frametitle{混成器拓展下的 X11 架构}
	
	这样,X11 的架构在混成器拓展下就变成了下图的样子:
	
	\digraph[scale=0.3]{x11composite}{
		rankdir=LR;
		splines=ortho;
		node [shape=box];
		xlib [label="xlib/XCB"];
		gtk [label="GTK"];
		qt [label="Qt"];
		compositor [label="Compositor", height=2];
		wm [label=<Window<br />Manager>];
		{rank=same; xlib; qt; gtk; compositor; wm;}
		pdf [label="Internal PDF Format"];
		qt_internal [label="Internal Format"];
		gtk_xpm [label="XPM"];
		qt_xpm [label="XPM"];
		xserver [label="Xserver", height=5];
		screen [label="Screen", height=8];
		gtk -> pdf [label="Cairo"];
		pdf -> gtk_xpm;
		gtk_xpm -> xserver [label="xlib/xcb"];
		qt -> qt_internal [label="QPaint"];
		qt_internal -> qt_xpm;
		qt_xpm -> xserver [label="xlib/xcb"];
		xlib -> xserver [label="xlib/xcb"];
		wm -> xserver;
		xserver -> wm;
		xserver -> compositor [label="NameWindowPixmap API"];
		compositor -> screen [label="XRender/OpenGL"];
		xserver -> screen [color=white];
		{ rank=max; screen; }
		gtk -> qt [color=white];
		qt -> xlib [color=white];
		wm -> xlib [color=white];
	}
	
\end{frame}

\begin{frame}
	\frametitle{那么输入事件呢?}
	有了混成器,我们现在可以重定向输出事件。\\
	那么,输入事件呢?\\
	输入事件可以分为两类,键盘事件,和鼠标事件。\\
	对于键盘事件,如果要处理,需要截取所有键盘事件,然后重新派发,水很深,所以大多数混成器都直接放弃了重定向键盘事件。\\
	对于鼠标事件,就更复杂了,所以混成器一般使用 XFixes 拓展提供的功能将上面提到的 OverlayWindow 设置成对鼠标事件透明,从而使得点击直接透传到下面的原有窗口。\\
	但是这也造成了严重的问题,就是一旦混成器绘制的图像与原有的不符,就会造成错误的结果。
	
\end{frame}
\begin{frame}
	\frametitle{那么输入事件呢?}
	所以混成器一般会有两个状态,一个状态保证窗口大小位置形状完全一致,也就是通常的状态。\\
	如果需要的话(例如有些桌面有一个应用切换界面),可以发生不一致,但是这时候所有鼠标事件都会被拦截,只保留混成器专门留下的(比如切换模式下点击窗口,或者点击专门的关闭窗口按钮)。\\
	注意这里只能有专门的关闭按钮,因为这时候鼠标输入是无法传递给窗口的。
	
\end{frame}

\section{我们为什么要有 Wayland}
\begin{frame}
	\frametitle{我们为什么要有 Wayland}
	面对这些问题,有人提出了一个观点:\\
	既然 X11 因为基础设计原因有着这样那样的问题,我们为什么不推倒重来呢,重新设计,避开 X11 所遇到的问题,从头开始。\\
	这就是 Wayland 了。\\
	那么,Wayland 都解决了什么问题呢?\\
	\begin{itemize}
		\item 同样的位图从 client 发送到 xserver,再从 xserver 发送到混成器,这是无谓的开销,正如 Daniels 所说:
				\begin{quote}
					and what's the X server? really bad IPC\footnote{\url{https://people.freedesktop.org/~daniels/lca2013-wayland-x11.pdf}}
				\end{quote}
		\item 只有像素的绘图原语,难以实现分数缩放和由 display server 控制的缩放,以及子像素抗锯齿
		\item 如上文所述,基本可以认为是没有的输入事件重定向
	\end{itemize}
	
\end{frame}

\section{Wayland 的架构}
\begin{frame}[fragile]
	\frametitle{Wayland 的架构}
	\digraph[scale=0.4]{wayland}{
		rankdir=LR;
		splines=ortho;
		node [shape=box];
		xlib [label="xlib/XCB"];
		gtk [label="GTK"];
		qt [label="Qt"];
		{rank=same; xlib; qt; gtk;}
		xwayland [label="XWayland"];
		compositor [label=<Wayland<br/>Compositor>, height=4];
		screen [label="Screen", height=2];
		keyboard [label="Keyboard", height=1.5];
		mouse [label="Mouse", height=1.5];
		{rank=same; screen; keyboard; mouse;}
		xlib -> xwayland [label="xlib/XCB"];
		xwayland -> compositor;
		gtk -> compositor [label="Cairo"];
		qt -> compositor [label="QPaint"];
		compositor -> screen [label="KMS & DRM by EGL"];
		keyboard -> compositor [label="evdev & libinput"];
		mouse -> compositor [label="evdev & libinput"];
		compositor -> gtk;
		compositor -> qt;
		compositor -> xwayland;
		xwayland -> xlib;
	}
	
\end{frame}

\section{Wayland 的罪与罚}
\begin{frame}
	\frametitle{Wayland 的罪与罚}
	
	\begin{itemize}
		\item 首先是老生常谈的,有许多应用尚不支持 Native Wayland,而且他们中的很多也不可能支持 Wayland 了
		\item 在 X11 下 Window 有一个 Class,可以用于识别窗口,由此可以实现很多有用的功能,比如窗口自动归类和对不同应用使用不同的输入法(Rime)和输入法状态(Fcitx5)
		\item 因为 Wayland 的协议设计缺陷,输入法不能获得准确的光标位置来绘制候选框,同样的,不能移动候选框位置
		\item 同样的,因为 Wayland 输入法协议有多个版本,如果应用,输入法和 Wayland 混成器支持的协议没有交集,会直接无法使用输入法\footnote{\url{https://gitlab.freedesktop.org/wayland/wayland-protocols/-/issues/39\#implementation-matrix}}
		\item 同时,Wayland 输入法协议要求输入法进程被 Compositor 启动
	\end{itemize}
	
\end{frame}

\begin{frame}[fragile]
	\frametitle{Wayland 下的输入法架构}
	\digraph[scale=0.3]{waylandinput}{
		rankdir=LR;
		splines=ortho;
		node [shape=box];
		qt_im_module [label=<Qt application<br/>using Qt IM Module>];
		gtk_im_module [label=<GTK application<br/>using GTK IM Module>];
		v2 [label=<Application<br/>using zwp_text_input_v2 protocol>];
		v3 [label=<Application<br/>using zwp_text_input_v3 protocol<br/>e.g. foot>];
		x11 [label="X11 application"];
		xwayland [label="XWayland"];
		dbus [label="DBUS Daemon", height=2];
		compositor [label=<Wayland<br/>Compositor>, height=2];
		im [label="Input Method", height=6.5];
		qt_im_module -> dbus [dir="both"];
		gtk_im_module -> dbus [dir="both"];
		dbus -> im [dir="both"];
		v2 -> compositor [dir="both", label="zwp_text_input_v2"];
		v3 -> compositor [dir="both", label="zwp_text_input_v3"];
		compositor -> im [dir="both", label=<zwp_input_method_v1<br/>or<br/>zwp_input_method_v2>];
		x11 -> xwayland [dir="both", label="XIM"];
		xwayland -> im [dir="both", label="XIM"];
		{ rank=same; v2; v3; x11; qt_im_module; gtk_im_module;}
		{ rank=same; compositor; dbus; xwayland;}
	}
\end{frame}

\begin{frame}[fragile]
	\frametitle{害群之马}
	\digraph[scale=0.4]{gnomeinput}{
		rankdir=LR;
		splines=ortho;
		node [shape=box];
		v2 [label=<Application<br/>using zwp_text_input_v2 protocol>];
		v3 [label=<Application<br/>using zwp_text_input_v3 protocol<br/>e.g. foot>];
		dbus [label="DBUS Daemon", height=2];
		compositor [label=<Mutter<br/>Wayland Compositor>, height=2];
		im [label="Input Method", height=2];
		v2 -> compositor [dir="both", style="dashed", label="zwp_text_input_v2"];
		v3 -> compositor [dir="both", label="zwp_text_input_v3"];
		compositor -> dbus [dir="both", label="IBus DBus protocol"];
		dbus -> im [dir="both", label="IBus DBus protocol"];
		{ rank=same; compositor; dbus;}
		{ rank=same; v2; v3; im;}
	}
\end{frame}

\section{讨论时间}
\begin{frame}
	\begin{center}
		\huge{讨论时间}
	\end{center}
\end{frame}
\begin{frame}
	\begin{center}
		\huge{谢谢大家}
	\end{center}
\end{frame}

\end{document}
