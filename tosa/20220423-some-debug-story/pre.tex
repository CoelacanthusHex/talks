\documentclass[UTF-8]{ctexbeamer}
\usetheme{Berkeley}
\usecolortheme{seahorse}

\usepackage{minted}
\usepackage{tikz}
\usetikzlibrary{arrows,positioning,calc,decorations.pathreplacing,fit}
\usepackage{xcolor}
\usepackage{hyperref}
\usepackage{textcomp}
\usepackage{amsmath}
\usepackage{amssymb}
\usepackage{unicode-math}
\usepackage{graphicx}
\usepackage[normalem]{ulem}
\usepackage{tabularx}
\usepackage{multirow}
\usepackage{chronosys}

\setminted{
	tabsize=4,
	frame=lines
}
\hypersetup{
	colorlinks,
	linkcolor=blue,
	urlcolor=blue,
	anchorcolor=blue,
	citecolor=blue,
	breaklinks=true,
	hyperfootnotes=true,
	unicode
}

\newcommand{\emptyline}{\vspace{\baselineskip}}

\usepackage{FiraMono}

%\setmonofont{Fira Mono}
\setmainfont{Noto Serif}
\setsansfont{Noto Sans}
\setCJKmainfont{Noto Serif CJK SC}
\setCJKsansfont{Noto Sans CJK SC}
\setCJKmonofont{Noto Sans CJK SC}

\title{我与 Firefox 的故事}
%\subtitle{}
\author{Coelacanthus}
%\institute{PLCT Arch RISC-V 小队}
\date{2022年4月21日}
%\logo{
%	\includegraphics[scale=0.75]{assets/Archlinux-icon.pdf}
%}

\setbeamertemplate{itemize items}[circle]
\setbeamertemplate{enumerate items}[default]
\beamertemplatenavigationsymbolsempty
\setbeamerfont{footnote}{size=\tiny}

%\urlstyle{tt}

\begin{document}

\frame{\titlepage}

\section{什么是火狐}
\begin{frame}

	你可以看 \url{https://www.bilibili.com/video/BV1fY41177mf}

\end{frame}

\section{我与火狐的故事}
\begin{frame}[fragile]
	\frametitle{问题的开始}
	
	\begin{itemize}[<+->]
		\item 一个需求:我想要寻找一个合适的等宽字体
		\item 后来,我找到了 Iosevka,但是,我不喜欢他的 19 种预设字形
		\item 然后,我发现 Iosevka 支持可以编译时控制使用哪个字形
	\end{itemize}
	
\end{frame}

\begin{frame}[fragile]
	\frametitle{问题的开始}

	\begin{itemize}[<+->]
		\item 然后,我发现 Iosevka 支持通过 OpenType Font Feature\footnote{\url{https://en.wikipedia.org/wiki/List_of_typographic_features}} 控制使用预设的某个字形
		\item 紧接着经过一番查找文档,我发现 fontconfig\footnote{\url{https://www.freedesktop.org/wiki/Software/fontconfig/}} 可以控制字体的 OpenType Font Feature
		\item 然后我开始尝试编写 fontconfig 配置
	\end{itemize}
	
\end{frame}

\begin{frame}[fragile]
	\frametitle{问题的开始}
	
	最初的配置长这个样子:
	\begin{minted}[fontsize=\tiny]{xml}
<?xml version='1.0'?>
<!DOCTYPE fontconfig SYSTEM 'fonts.dtd'>
<fontconfig>
  <match target="pattern">
    <test name="family" compare="eq" ignore-blanks="true">
      <string>Iosevka</string>
    </test>
    <edit name="fontfeatures" mode="append">
      <string>ss10 on</string> <!-- using SS10 variant shape -->
      <string>cv49 1</string>  <!-- straight y -->
      <string>cv83 2</string>  <!-- underline, not too high, not too low -->
     </edit>
  </match>
</fontconfig>
	\end{minted}
	
\end{frame}

\begin{frame}[fragile]
	\frametitle{问题的开始}
	
	\begin{itemize}[<+->]
		\item 我使用 pango-view 进行测试,发现他工作的很好……
		\item 直到……
		\item 直到我开始使用 Qt 编写的 Konsole 和 自己进行字体选择的 Firefox
		\item 他不工作!
	\end{itemize}
	
\end{frame}

\begin{frame}
	\frametitle{What happened?}
	
	Qt: \href{https://bugreports.qt.io/browse/QTBUG-78645}{Support OpenType font features through fontconfig} opened at 2019-09-20 \\
	\emptyline \\
	Firefox: No bug report,所以我开了一个\\ \href{https://bugzilla.mozilla.org/show_bug.cgi?id=1744765}{Add font features config from user's fontconfig}
	
\end{frame}

\begin{frame}
	\frametitle{What happened?}
	
	但是等了\sout{很久}三个月也没人去做,于是我决定自己动手
	
\end{frame}

\begin{frame}[fragile]
	\frametitle{我是怎么做的}
	
	\begin{itemize}[<+->]
		\item 首先把火狐的代码拖下来
		\item 然后去找这部分代码在什么地方:\verb|gfx/thebes/|
		\item 然后理清调用链条(火狐大量使用继承找起来还是有点麻烦的,推荐配合好用的代码跳转工具)
		\item 最后寻找插入代码的位置
		\item 改完,编译,测试,提交,收工
		\item 这就完了?
	\end{itemize}
	
\end{frame}

\begin{frame}[fragile]
	\frametitle{不,还没结束!}
	
	\begin{itemize}[<+->]
		\item 提交之后,review 机器人迅速指出了第一个问题:把 unsiged int 转换成 signed int 的行为是实现定义的
		\item 然后,jfkthame 指出了第二个和第三个问题:我判断条件写反了,并且我没有必要创建一个新数组,我可以复用原本为 CSS 的 font-feature-settings\footnote{\url{https://developer.mozilla.org/en-US/docs/Web/CSS/font-feature-settings}} 设计的 mFeatureSettings 数组
	\end{itemize}
	
\end{frame}

\begin{frame}[fragile]
	\frametitle{不,还没结束!}
	
	\begin{itemize}[<+->]
		\item 重新修改并提交之后,jfkthame 又指出了两个问题:Firefox 代码风格不使用 \verb|this->| 和 C-style cast,直接使用初始化列表
		\item 然后,他又指出,对网络字体附加用户本地配置可能不是预期的行为,也许应当只对本地字体应用设置
	\end{itemize}
	
\end{frame}

\begin{frame}[fragile]
	\frametitle{另一个故事}
	
	\begin{itemize}[<+->]
		\item 有一天,我浏览\href{https://asaba.sakuragawa.moe/2021/07/%e4%bf%ae%e5%be%a9-fedora-gnu-linux-%e7%b3%bb%e7%b5%b1%e4%b8%8b%e7%9a%84%e9%8d%b5%e7%9b%a4%e5%8a%9f%e8%83%bd%e5%8d%80%ef%bc%88f-%e5%8d%80%ef%bc%89%e6%8c%89%e9%8d%b5/}{一个 zh-HK 的网页},突然发现代码块使用的字体不是等宽的
		\item 经过仔细排查,发现仅在我的 locale 为 zh-CN 且页面语言为 zh-HK 和 zh-TW 时发生
		\item 于是我去开了个 Bug report: \url{https://bugzilla.mozilla.org/show_bug.cgi?id=1756400}
		\item 后来发现,当我在 fontconfig 里重复 prepend 一个字体的时候,会触发这个问题
		\item 但是我的重复 prepend 只针对 zh-CN locale 啊?
	\end{itemize}
	
\end{frame}

\begin{frame}[fragile]
	\frametitle{另一个故事}
	
	\begin{itemize}[<+->]
		\item 然后,发现了一个问题
		\item 火狐查询字体的时候,传的是系统 locale,而不是页面 locale!
		\item 维护者表示尽管关键问题不在这里,但是他仍然应该被修复
		\item 所以火狐现在变成了一个错误 prepend 检测器,比如 \href{https://bugzilla.mozilla.org/show_bug.cgi?id=1763175}{Firefox font regression in Arch Linux (KDE plasma 5.24.4)}
	\end{itemize}
	
\end{frame}

\begin{frame}
	\begin{center}
		\huge{谢谢大家}
	\end{center}

\end{frame}

\end{document}
